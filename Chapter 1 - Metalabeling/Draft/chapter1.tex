\documentclass[a4paper]{article}

\usepackage[english]{babel}
\usepackage[utf8]{inputenc}
\usepackage{fullpage}
\usepackage{amsmath}
\usepackage{graphicx}
\usepackage[colorinlistoftodos]{todonotes}
\usepackage{hyperref}
\usepackage{amssymb}
\usepackage{outline} 
\usepackage{pmgraph} 
\usepackage[normalem]{ulem}
\usepackage{graphicx}  % Lets you insertPDF images
\usepackage{svg}  % Lets you insert svg images
\usepackage{verbatim}
\usepackage{afterpage}  % new pages

% Notes
\usepackage{todonotes}

% Colored text
\usepackage{xcolor}

% Bold mathematical expressions
\usepackage{fixmath}

% Confusion Matrix
\usepackage{array}
\usepackage{multirow}
\newcommand\MyBox[2]{
  \fbox{\lower0.75cm
    \vbox to 1.7cm{\vfil
      \hbox to 1.7cm{\centering \hfil\parbox{0.5cm}{#1\\#2}\hfil}
      \vfil}%
  }%
}

% Lorem Ipsum
\usepackage{lipsum}

% Tables
\usepackage{array}
\newcolumntype{C}[1]{>{\centering\arraybackslash}m{#1}}

\newcommand{\homeCOne}{../../Chapter 1 - Metalabeling/Draft}
\newcommand{\homeCTwo}{../../Chapter 2 - FracDiff/Draft}

% ____ Bibliography ____
\usepackage{natbib} %Bibliography.
% \setcitestyle{numbers} %Cite as numbers or author-year.
\bibliographystyle{plain} %Reference style.
% ______________________

\begin{document}
\begin{titlepage}
  \centering
  \sc \LARGE

  \rule{\textwidth}{1.6pt}\vspace*{-\baselineskip}\vspace*{2pt}
  \rule{\textwidth}{0.4pt}
  \vspace{-6mm}

  \textbf{Draft - Chapter 1: Meta-labeling}\\[1.25ex]

  \vspace{-3.5mm}
  \rule{\textwidth}{0.4pt}\vspace*{-\baselineskip}\vspace*{3.2pt}
  \rule{\textwidth}{1.6pt}
  \vfill

  \Large
 \centering{Author: Guillermo Creus Botella} \\

  \Large
  Supervisor: Daniel P. Palomar
  \vfill

  \includegraphics[height=3.5cm]{img/hkust.png} \hspace{2.5cm}
  \includegraphics[height=3.5cm]{img/upc.png}
  \vspace{1cm}

  \textit{The Hong Kong University of Science and Technology} (HKUST)\\
  \vspace{1cm}
  \textit{Polytechnic University of Catalonia - BarcelonaTech} (UPC)\\
  \vfill

\end{titlepage}

\pagenumbering{arabic}
\null
\thispagestyle{empty}
\addtocounter{page}{-3}
\newpage

\thispagestyle{empty}
\tableofcontents

\newpage\null\thispagestyle{empty}\newpage

\section{Introduction}
\label{sec:intro}
Labeling is a well-known area in Machine Learning. It consists of 
gathering a matrix $X$, known as features, whose rows are the 
observations. Once a label $y$ is assigned to every observation $X_{i,
\cdot}$ the main goal is to give a prediction $\hat{y}_i$. Whenever $y 
\in I$ s.t. $|I| = 2$, the problem can be referred as a binary 
classification, which is what this chapter will focus on. \\

When it comes to Finance, labeling is not as straightforward as a 
conventional case, i.e. predicting deterministic events; yes/no 
traffic light in the picture, yes/no happy face, etc. Since one is 
working with returns, one needs to determine whether a positive 
(negative) outcome will happen (or not) in a determined time horizon. 
\\

The investment literature has tried to label observations using what 
Marcos López de Prado (MLDP) \cite{AdvFML} defines as \textit{The 
Fixed-Time Horizon Method}:

\begin{equation}
	y_i =
    \begin{cases}
      -1 & \text{if}\ r_{t_{i, 0}, t_{i, 0} + h} < \tau \\
      \hfill 0 & \text{if}\ |r_{t_{i, 0}, t_{i, 0} + h}| \leq \tau \\
      \hfill 1 & \text{if}\ r_{t_{i, 0}, t_{i, 0} + h} > \tau
    \end{cases}
\end{equation}

Where $r_{t_{i, 0}, t_{i, 0} + h}$ is the linear return from time 
$t_{i, 0}$ to $t_{i, 0} + h$ ($h$ is a time bar that can be days, 
months, etc.).\\

The problem with computing labels with a fixed threshold $\tau$ is 
that volatility, $\sigma$, changes overtime and should be updated 
regularly. Apart from that, restricting the model with a fixed $h$ is 
not optimal at all since more flexible ways can be implemented.

\section{Motivation}
\label{sec:motiv}
A portfolio of $N$ securities is defined by the weights, $\textbf{w} 
\in \mathbb{R}^N$ it gives to every instrument. In order to minimize 
the volatility, the Global Minimum Variance portfolio (\textbf{GMVP}) 
is defined as:

\begin{equation}
	\begin{aligned}
		\min_{w} \quad & w^{T} \Sigma w\\
		\textrm{subject to} \quad & w \geq 0\\
 		 & \sum_{i = 1}^{N} w_i = 1    \\
	\end{aligned}
\end{equation}

Where $\Sigma$ is the Covariance matrix of returns. If they are 
linear, then they are defined as: $R_{t}^{i} = \frac{p_{t}^{i}}
{p_{t-1}^{i}} - 1$.\\

At this point, the portfolio budget is defined as $B$ and the amount 
of money in asset $i$ is simply $B w_i$. Now, the portfolio returns 
can be computed as:

\begin{equation}
	R_{t}^{P} = \frac{\sum_{i = 1}^{N} B w_i \cdot (1 + R_{t}^{i})}{B} - 1 =
	\sum_{i = 1}^{N} (w_i + w_i \cdot R_{t}^{i}) - 1 =
	\textbf{w} \cdot \textbf{$R_{t}$}
\end{equation}

Note that if $B w_i$ is the initial wealth of asset $i$ at time $t-1$, 
then the final wealth is:

\begin{equation}
	B w_i \cdot \frac{p_{t}^{i}}{p_{t-1}^{i}} = B w_i \cdot (1 + R_{t}^{i})
\end{equation}

With all these concepts laid out, it is appropriate to talk about a 
single time series. In other words, the portfolio indirectly 
transforms a multivariate time series into a univariate one. 
Consequently, from now on, only the portfolio time series will be 
analyzed.\\ 

As it can be seen in figure \ref{fig:GMVP_fall}, the falls in late 
2018 and early 2020 are a huge setback as far as the final wealth is 
concerned. Going back to section \ref{sec:intro}, the general idea is 
to label these periods accordingly so that these drawdowns can be 
predicted. \\

\begin{figure}[htbp]
	\centering
	\includegraphics[scale=.4]{"\homeCOne/img/GMVP_fall"}
	\caption{GMVP portfolio prices}
	\label{fig:GMVP_fall}
\end{figure}

\section{Toy Project}
\label{sec:toyProject}
Before applying Meta-labeling to financial data, an experiment has 
been designed in order to explain how meta-labeling works with 
synthetic data. Labels of side (long/short) of investments will be 
created in a way that a designated set of features are responsible for 
it. That is, features that have predicting power. The desired output 
is to predict whether one should open a position or not, and the side 
of it. This will be done in a sequential manner, with a primary and 
secondary model.\\

The primary Model will predict the side of the investment, and the 
secondary model will decide if the primary model was right. To be 
specific, the primary model will tell you to open a position (positive 
in the binary classification context) with a given side, and the 
secondary model will decide if it is a false or true positive.\\

Meta-labeling should be used when one wants to achieve higher 
F1-Scores (harmonic mean of precision and recall), because it lowers 
the high recall from the primary model while getting a much higher 
precision, thus boosting the F1-Score.\\

As for the data division, the primary model will use a set of features 
different than the secondary model ones. The latter will have some 
designated features plus the prediction from the primary model That 
way, with the new information and the prediction, the secondary model 
will evaluate if the primary model made a correct decision.

\subsection{Data}
The data for this Toy Project will consist on 1000 observations of the 
following data points:

\begin{itemize}

	\item \textbf{Features:} $\textbf{X}_{k,i} \sim 
	N(\mu_i,\ \sigma^2)\ \text{for } i \in \{1,\ \ldots,\ 5\},\ 
	k \in \{1,\ \ldots,\ 1000\}$
	
	\item $\omega_k = \textbf{sigmoid} \left( \alpha + 
	\sum_{i = 1}^{5} \textbf{X}_{k,i} \cdot \beta_i + \epsilon_k 
	\right)$ where $\epsilon_k \sim N(0,\ \sigma_{\epsilon}^2)$ 
	and $\textbf{sigmoid}(z) = \frac{1}{1 + e^{-z}}$
	
	\vspace{.1cm}

	Where: $\alpha = -0.5970$, \\
	$\mathbold{\beta}=[-0.7862,\ 0.7695,\ -1.3740,\  0.6722,\ -0.4536]
	$,\\
	$\mathbold{\mu} = [-0.3110,\ -0.1157,\ 0.0316,\ 0.3210,\ -0.5933]
	$,\\
	$\sigma = 0.5$ and $\sigma_\epsilon \in 
	\{0,\ 0.1,\ \ldots,\ 2.9,\ 3\}$\\
	
	
	\item \textbf{Labels:} The side will be designated using 
	$\omega_k$, which implies that all \textbf{5 features} have an 
	effect on the correct side/investment.
	\begin{equation*}
		y_k^{\text{M1}} =
	    \begin{cases}
	      -1 & \text{if}\ \omega_k < 0.5 \\
	      \hfill 1 & \text{otherwise} 
	    \end{cases}
	\end{equation*}
\end{itemize}

Note that, in order to avoid unbalanced labels, the following 
constraint has been applied: 
$0 = \alpha + \mathbold{\beta \cdot \mu}$. 
That way, the expected value of $\alpha + \sum_{i = 1}^{5} 
\textbf{X}_{k,i} \cdot \beta_i + \epsilon_k$ is 0 ($\textbf{sigmoid}
(0) = 0.5$).\\

By generating data this way, there are 5 explanatory variables that 
are responsible for the position opening. Lastly, the labels of the 
secondary model will be a 1 whenever the primary model gave a correct 
prediction of the side and 0 otherwise. To be specific:

\begin{equation*}
	y_k^{\text{M2}} =
    \begin{cases}
      1 & \text{if}\ \hat{y}^{\text{M1}}_k = 
      y^{\text{M1}}= -1\ \text{or}\ 
      \hat{y}^{\text{M1}}_k = 
      y^{\text{M1}}= 1\\
      \hfill 0 & \text{otherwise} 
    \end{cases}
\end{equation*}

\subsection{Models}
Since this section intends to give a general overview of the way meta-
labeling works, the models will use different features in order to 
simulate relative abundance or scarcity of data. That is:

\begin{itemize}
	\item M1: $\textbf{X}_1$ \hfill 
	($N_\text{M1} = 1$) 	\hspace{0.5\textwidth} \\
	M2: $\hat{y}^{\text{M1}},\ \textbf{X}_2,\ \ldots,\ \textbf{X}_5$ 
	\hfill ($N_\text{M2} = 5$) \hspace{0.5\textwidth}
	
	\item M1: $\textbf{X}_1,\ \textbf{X}_2$ \hfill
	($N_\text{M1} = 2$) \hspace{0.5\textwidth} \\
	M2: $\hat{y}^{\text{M1}},\ \textbf{X}_3,\ \textbf{X}_4,\ 
	\textbf{X}_5$ 
	\hfill ($N_\text{M2} = 4$) \hspace{0.5\textwidth}
	
	\item M1: $\textbf{X}_1,\ \textbf{X}_2,\ \textbf{X}_3$ \hfill 
	($N_\text{M1} = 3$) \hspace{0.5\textwidth} \\
	M2: $\hat{y}^{\text{M1}},\ \textbf{X}_4,\ \textbf{X}_5$ 
	\hfill ($N_\text{M2} = 3$) \hspace{0.5\textwidth}
	
	\item M1: $\textbf{X}_1,\ \ldots,\ \textbf{X}_4$ \hfill
	($N_\text{M1} = 4$) \hspace{0.5\textwidth} \\
	M2: $\hat{y}^{\text{M1}},\ \textbf{X}_5$ 
	\hfill ($N_\text{M2} = 2$) \hspace{0.5\textwidth} \\
\end{itemize}

\textbf{Primary Model (M1):} It will use $y^{\text{M1}}$ as labels. 
The underlying model used is a single layer neural network with a 
sigmoid activation function.\\

\vspace{.1cm}
\noindent
\textbf{Secondary Model (M2):} It will use $y^{\text{M2}}$ as labels. 
The underlying model used is a neural network with a hidden layer (25 
units - leaky ReLU) and an output unit with a sigmoid activation 
function.\\

\vspace{.1cm}
\noindent
\textbf{Meta Model (M1 + M2):} This model is the combination of the 
previous two models. It will decide to open a position with side 
$\pm 1$ if the M1 predicts a side $\pm 1$ and the M2 predicts a 1 
(i.e. M1 is right). In contrast, if M2 predicts a 0 (M1 is wrong), 
then the Meta Model will not open a position.\\

The obvious question regarding the meta model is why cannot one train 
a single model instead of dividing the data between models. Although 
the single model will achieve higher precision scores, it will defeat 
the purpose of meta-labeling. In other words, one of the strengths of 
meta-labeling is being able to integrate ML into a 
fundamental/technical analysis approach or a model already up and 
running. That is, the secondary model will act as an exogenous model 
and not something that could have been designed from the start.\\

Lastly, the models will use 80\% of data as in-sample and 20\% of data 
as out-of-sample. The former will be further divided into training 
(80\%) and validation (20\%) so as to avoid over fitting.

\subsection{Results}
Before evaluating the results, a confusion matrix of this project, the 
``positive'' and ``negative'' outcomes should be defined:

\begin{itemize}
	\item \textbf{1:} Open a position
	\item \textbf{0:} Do not open a position
\end{itemize}

This gives way to:

\begin{itemize}
	\item \textbf{TP:} Opened a position that was profitable. 
	\item \textbf{FP:} Opened a losing position.
	\item \textbf{TN:} Did not open a position that was going to be 
	unprofitable.
	\item \textbf{FN:} Took a pass at opening a position that was 
	going to be profitable. 
	wrong.
\end{itemize}

On the other hand, the metrics that will be used to evaluate the 
models will be the following:
\begin{itemize}
	\item \textbf{Recall} $ = \frac{\text{TP}}{\text{TP} + \text{FN}}$
	
	\item \textbf{Precision} 
	$ = \frac{\text{TP}}{\text{TP} + \text{FP}}$
	
	\item \textbf{F1 - Score} 
	$ = \frac{2}{\text{recall}^{-1} + \text{precision}^{-1}}$
\end{itemize}

\subsubsection{Example}
This subsection will exemplify what meta-labeling does in terms of 
relabeling false positives as true negatives, confusion matrices and 
metrics. The hyper parameters chosen are:

\begin{itemize}
	\item $\sigma_\epsilon = 0.3$
	\item $N_{\text{M1}} = 2$
	\item $N_{\text{M2}} = 4$
\end{itemize}

As for the results, the confusion matrices of the primary model and 
Meta Model are shown in figures \ref{toyProjectConfusionMatrixM1} and 
\ref{toyProjectConfusionMatrixMM}. The primary model (figure
\ref{toyProjectConfusionMatrixM1}) only predicts opening positions, 
i.e., it does not have the ability to pass. The Meta Model (figure
\ref{toyProjectConfusionMatrixMM}) corrects more FP than TP.\\

\begin{figure}[htbp]
	\centering
	\scalebox{.8}{
	\renewcommand\arraystretch{1.5}
	\setlength\tabcolsep{0pt}
	\begin{tabular}{c >{\bfseries}r @{\hspace{0.7em}}c 
	@{\hspace{0.4em}}c @{\hspace{0.7em}}l}
	  \multirow{10}{*}{\parbox{1.1cm}{\bfseries\raggedleft Actual\\ 
	  value}} & 
	    & \multicolumn{2}{c}{\bfseries Prediction outcome} & \\
	  & & \bfseries 1 & \bfseries 0 & \bfseries Total \\
	  & 1 & \MyBox{TP}{143} & \MyBox{FN}{0} & 143 \\[2.4em]
	  & 0 & \MyBox{FP}{57} & \MyBox{TN}{0} & 57 \\
	  & Total & 200 & 0 &
	\end{tabular}}
	\caption{Confusion Matrix - Primary Model (Test)}
	\label{toyProjectConfusionMatrixM1}
\end{figure}

\begin{figure}[htbp]
	\centering
	\scalebox{0.8}{
	\renewcommand\arraystretch{1.5}
	\setlength\tabcolsep{0pt}
	\begin{tabular}{c >{\bfseries}r @{\hspace{0.7em}}c @{\hspace{0.4em}}c @{\hspace{0.7em}}l}
	  \multirow{10}{*}{\parbox{1.1cm}{\bfseries\raggedleft Actual\\ value}} & 
	    & \multicolumn{2}{c}{\bfseries Prediction outcome} & \\
	  & & \bfseries 1 & \bfseries 0 & \bfseries Total \\
	  & 1 & \MyBox{TP}{136} & \MyBox{FN}{7} & 143 \\[2.4em]
	  & 0 & \MyBox{FP}{41} & \MyBox{TN}{16} & 57 \\
	  & Total & 177 & 23 &
	\end{tabular}}
	\caption{Confusion Matrix - Meta Model (Test)}
	\label{toyProjectConfusionMatrixMM}
\end{figure}

\begin{figure}[htbp]
\centering
	\includegraphics[scale=.45]{"\homeCOne/img/toyProjectMetrics"}
	\caption{Toy Project - Metrics of example (Test)}
	\label{fig:toyProjectMetrics}
\end{figure}

			
\begin{table}[htbp]
\centering
	\caption{Toy Project Metrics (Test)}
	\label{tab:toyProjectResults}
	\begin{tabular}{ |C{2.5cm}|C{2cm}|C{2cm}|C{2cm}|}
		\hline
		Model & F1-Score & Precision & Recall\\
		\hline
		Primary Model & 0.8338 & 0.7150 & 1.0000\\ 
		Meta Model & 0.8500 & 0.7684 & 0.9510\\
		\hline
	\end{tabular}
\end{table}	

The metrics obtained are shown in table \ref{tab:toyProjectResults} 
and figure \ref{fig:toyProjectMetrics}. As it can be implied from the 
confusion matrices, the recall has gone down but the precision and F1-
score have gone up. In particular, the F1-Score has gone up by 2\%, 
from 0.8338 to 0.85. This increase, even though on modest scale, is 
what meta-labeling was supposed to do.

\subsubsection{Precision}
\label{sec:toyProjectPrecision}
In figures \ref{fig:precisionN1} to \ref{fig:precisionN4}, the 
precision has been plotted for every $\sigma_{\epsilon}$. Whenever 
$N_{\text{M1}} \geq 3$, the meta model fails to improve the precision. 
This could be attributed to the primary model having the majority of 
the information, so the secondary model is not able to improve in the 
predictions given by the primary model.\\

On the other hand, if the primary model does not have a full picture 
of the information ($N_{\text{M1}} \leq 2$), the model is similar to 
the random model, which has a precision $= 0.5$. That way, the 
secondary model has more room to improve. To be more specific, if 
$\sigma_\epsilon \in [0,\ 0.7]\ \bigcup \ [2.6,\ 3]$, i.e. the 
observations have low/high signal-to-noise ratio (high/low 
$\sigma_\epsilon$), the meta model performs better.

\begin{figure}[htbp]
\centering
	\begin{minipage}{.5\textwidth}
	\centering
		\includegraphics[scale=.25]{"\homeCOne/img/toyProject/precisionN1"}
	  	\caption{Precision (Test) - $N_{\text{M1}} = 1$}
	  	\label{fig:precisionN1}
	\end{minipage}%
	\begin{minipage}{.5\textwidth}
	\centering
		\includegraphics[scale=.25]{"\homeCOne/img/toyProject/precisionN2"}
		\caption{Precision (Test) - $N_{\text{M1}} = 2$}
		\label{fig:precisionN2}
	\end{minipage}

	\vspace{.5cm}

	\begin{minipage}{.5\textwidth}
	\centering
		\includegraphics[scale=.25]{"\homeCOne/img/toyProject/precisionN3"}
		\caption{Precision (Test) - $N_{\text{M1}} = 3$}
		\label{fig:precisionN3}
	\end{minipage}%
	\begin{minipage}{.5\textwidth}
	\centering
		\includegraphics[scale=.25]{"\homeCOne/img/toyProject/precisionN4"}
		\caption{Precision (Test) - $N_{\text{M1}} = 4$}
		\label{fig:precisionN4}
	\end{minipage}
\end{figure}

\subsubsection{Recall}
\label{sec:toyProjectRecall}
In figures \ref{fig:recallN1} to \ref{fig:recallN4}, the recall has 
been plotted for every $\sigma_\epsilon$. Before anything, observe 
that the primary model, independently of $\sigma_\epsilon$ and 
$N_{\text{M1}}$, always has a recall of 1. That is because the primary 
model does not have the ability to pass on a position. In other words, 
it always spits out a side, so in any event, a position is always 
opened (it only predicts 1's).\\

That being said, as in section \ref{sec:toyProjectPrecision}, note 
that the meta model does not do a thing whenever $N_{\text{M1}} \geq 
3$. Additionally, if $\sigma_\epsilon \in [0,\ 0.7]$, then it is 
observed that for $N_{\text{M1}} \leq 2$ the recall falls 
considerably, so one can imply that if $\sigma_\epsilon$ is close to 
0, then the recall falls due to the secondary model filtering trades.

\begin{figure}[htbp]
\centering
	\begin{minipage}{.5\textwidth}
	\centering
		\includegraphics[scale=.25]{"\homeCOne/img/toyProject/recallN1"}
	  	\caption{Recall (Test) - $N_{\text{M1}} = 1$}
	  	\label{fig:recallN1}
	\end{minipage}%
	\begin{minipage}{.5\textwidth}
	\centering
		\includegraphics[scale=.25]{"\homeCOne/img/toyProject/recallN2"}
		\caption{Recall (Test) - $N_{\text{M1}} = 2$}
		\label{fig:recallN2}
	\end{minipage}

	\vspace{.5cm}

	\begin{minipage}{.5\textwidth}
	\centering
		\includegraphics[scale=.25]{"\homeCOne/img/toyProject/recallN3"}
		\caption{Recall (Test) - $N_{\text{M1}} = 3$}
		\label{fig:recallN3}
	\end{minipage}%
	\begin{minipage}{.5\textwidth}
	\centering
		\includegraphics[scale=.25]{"\homeCOne/img/toyProject/recallN4"}
		\caption{Recall (Test) - $N_{\text{M1}} = 4$}
		\label{fig:recallN4}
	\end{minipage}
\end{figure}

\subsubsection{F1-Score}
\label{sec:toyProjectF1Score}
In figures \ref{fig:F1ScoreN1} to \ref{fig:F1ScoreN4}, the F1-Score 
has been plotted for every $\sigma_\epsilon$. As it has been seen in 
\ref{sec:toyProjectPrecision} and \ref{sec:toyProjectRecall}, if 
$N_{\text{M1}} \geq 3$, the performance of the primary and meta model 
is indistinguishable. Furthermore, if $N_{\text{M1}} \leq 2$, the 
F1-Score slightly improved when $\sigma_\epsilon$ was low 
($\sigma_\epsilon \leq 0.6$).

\begin{figure}[htbp]
\centering
	\begin{minipage}{.5\textwidth}
	\centering
		\includegraphics[scale=.25]{"\homeCOne/img/toyProject/F1ScoreN1"}
	  	\caption{F1-Score (Test) - $N_{\text{M1}} = 1$}
	  	\label{fig:F1ScoreN1}
	\end{minipage}%
	\begin{minipage}{.5\textwidth}
	\centering
		\includegraphics[scale=.25]{"\homeCOne/img/toyProject/F1ScoreN2"}
		\caption{F1-Score (Test) - $N_{\text{M1}} = 2$}
		\label{fig:F1ScoreN2}
	\end{minipage}

	\vspace{.5cm}

	\begin{minipage}{.5\textwidth}
	\centering
		\includegraphics[scale=.25]{"\homeCOne/img/toyProject/F1ScoreN3"}
		\caption{F1-Score (Test) - $N_{\text{M1}} = 3$}
		\label{fig:F1ScoreN3}
	\end{minipage}%
	\begin{minipage}{.5\textwidth}
	\centering
		\includegraphics[scale=.25]{"\homeCOne/img/toyProject/F1ScoreN4"}
		\caption{F1-Score (Test) - $N_{\text{M1}} = 4$}
		\label{fig:F1ScoreN4}
	\end{minipage}
\end{figure}

\subsection{Conclusions}
In conclusion, the meta-labeling needs a specific setting so as to 
deliver better results. In fact, two situations have been identified:

\begin{enumerate}
	\item M1 performs poorly (low precision) and/or has few features
	with predicting 	power (in this case, low $N_{\text{M1}}$).
	\item M1 has the majority of the features with high predicting 
	power (in this case, high $N_{\text{M1}}$), i.e., it is already a 
	\textit{good} model.
\end{enumerate}

In situation 1, the meta model will perform better, in an F1-Score 
sense, whenever $\sigma_\epsilon$ is low and $N_{\text{M2}}$ is high. 
To put it another way, if the observations do not have a lot of noise, 
then the secondary model is able to correct the bad performance of the 
primary model. Note that since the 5 features are shared, whenever the 
primary model has few features with predicting power then the 
secondary model has a lot of predicting power.\\

In situation 2, M1 is already a decent model (recall = 1 and high 
precision when $\sigma_\epsilon$ is low). Consequently, it is very 
difficult to improve its performance. In fact, the F1-Score of the 
meta model is identical to the one of M1 because the secondary model 
fails to introduce new information (low $N_{\text{M2}}$).\\

The next step will be to try this methodology in financial data, which 
will be more challenging, since the synthetic data was designed to be 
somewhat predictable, even though it had Gaussian white noise.

\section{Data}
The universe of stocks considered are the ones that have been part of 
the S\&P500 in the period considered; 01-01-2000 to 09-01-2020. The 
stocks that have missing values or have gone bankrupt have been 
dropped out of the dataset. \\

The portfolio is not a proxy for the S\&P500 GMVP since the portfolio 
considered presents look-ahead bias. That is, if one were to compute 
day-to-day the GMVP portfolio, one would not obtain the same results 
since the \textit{losers} (companies that have gone bankrupt) have 
been removed, and the \textit{winners} (future constituents of the 
index) have been added. Nonetheless, this chapter has not been 
designed in order to beat the index. In fact, this chapter has been 
designed with a focus on delivering better risk-adjusted results, 
i.e., higher Sharpe Ratios. \\

The tickers of the stocks considered are shown in tables 
\ref{table:tickers} and \ref{table:tickersCont} (pages
\pageref{table:tickers} and \pageref{table:tickersCont} resp.).

\subsection{Removal of outliers}
\label{sec:removalOutliers}
In an attempt to reduce the influence of outliers on the model, the 
data has been cleaned using the R package imputeFin \cite{imputeFin}.
 In particular, the function \textbf{impute\_AR1\_t} has been used 
 with the following parameters:

\begin{itemize}
	\item \textbf{outlier\_prob\_th} $= 0.005$ - Threshold of 
	probability of observation to declare an outlier.
	\item \textbf{remove\_outliers} = TRUE
\end{itemize}

In order to apply the function, data has been divided into 10 chunks. 
Then, log-prices will be fitted an autoregressive model of order 1 
where the residuals follow a t-Student distribution.

% \todo[shadow, color=gray!40]{How many df? Least squares?} with 
% \ldots degrees of freedom:}

\begin{equation*}
	\log (p_t) = \phi_0 + \phi_1 \cdot \log (p_{t-1}) + \epsilon_t
\end{equation*}

The AR(1) model together with \textbf{outlier\_prob\_th} identifies 
the outliers  and imputes values so as to preserve statistical 
parameters in the time series. In figure \ref{fig:removalOutlierGMVP} 
one can see the difference between the original log-price time series 
(GMVP Outliers) and the imputed one (GMVP Imputed).

\begin{figure}[htbp]
	\centering
	\includegraphics[scale=.4]{"\homeCOne/img/removalOutliersGMVP"}
	\caption{Imputed log-price time series of the GMVP}
	\label{fig:removalOutlierGMVP}
\end{figure}

\subsection{Division of Data}
The data is divided as following (see figure \ref{fig:dataDivision}):

\vspace{.1cm}

\textbf{In-Sample:}
\begin{itemize}
	\item Train (2001-08-06/2013-10-18): Data set that will be used to 
	train the ML models.
	
	\item Validation (2013-10-21/2016-11-04): Data set to assess the 
	performance of the ML model. It will be used to tune the 
	hyper parameters.
\end{itemize}

\textbf{Out-of-sample:}
\begin{itemize}
	\item Test (2016-11-07/2020-08-31): In this data set, the model 
	will generate \textit{out-of-sample} predictions and the 
	performance will be close to real since data will be seen for the 
	first time.
\end{itemize}

\begin{figure}[htbp]
	\centering
	\includegraphics[width=.9\textwidth]{"\homeCOne/img/dataDivision"}
	\caption{Data Division}
	\label{fig:dataDivision}
\end{figure}	

\section{Labeling (Triple barrier)}
\label{sec:tripleBarrier}
In section \ref{sec:intro}, \textit{The Fixed-Time Horizon Method} was 
discussed. In an attempt to improve the previous method, MLDP defined 
the triple barrier method \cite{AdvFML}. It consists in:

\begin{itemize}
	\item \textbf{Horizontal barriers:} Dynamic levels that depend on 
	the 10 day rolling volatility.
	\item \textbf{Vertical barrier:} Set as a fixed time horizon. In 
	this case, 10 days.
\end{itemize}

Then, $t_{i,0}$ is the start day, $t_{i,0} + h$ the vertical barrier 
and $p_{t_{i,0}}$ the price at time $t_{i,0}$. Also, 
$p_{t_{i,0}} + \text{pt} \cdot \sigma_{t_{i,0}}$ is the upper 
horizontal barrier and $p_{t_{i,0}} - \text{sl} \cdot 
\sigma_{t_{i,0}}$ the lower horizontal barrier, where 
$\sigma_{t_{i,0}}^2 = \frac{\sum_{k=0}^{19}(r_{t_{i,0}-k} - \bar{r}
_{[t_{i,0}-19, t_{i,0}]})^2}{20 - 1}$. Lastly, hyper-parameters
\textbf{pt} and \textbf{sl} both have been set to 
\textbf{2}.\\

With these variables, the condition to exit a position that was opened 
at time $t_{i,0}$ is hitting a barrier. Note that exiting is assured 
since at most the position will be open for 10 days (vertical 
barrier).\\

When a barrier is hit ($t_{i,1}$), the return is computed as:

\begin{equation}
	r_i = \left[ (1 - \text{tc})^2 \cdot 
	\frac{p_{t_{i,1}}}{p_{t_{i,0}}} \right]- 1
\end{equation}

Where \textbf{tc} is the one-way transaction cost, which has been fixed at \textbf{5 bps} (0.05\%).\\

Combining everything, the observations are labeled as:
\begin{equation}
	y_i =
    \begin{cases}
      1 & \text{if}\ r_i > 0 \\
      0 & \text{otherwise}
    \end{cases}
\end{equation}

In figure \ref{fig:tripleBarrierSymmetric}, an observation
\textbf{labeled 1} can be seen. In early January 2002 the barriers 
start and the price touched the upper horizontal barrier first, hence, 
obtaining a positive return.

\begin{figure}[htbp]
	\centering
	\includegraphics[width=.75\textwidth]
	{"\homeCOne/img/tripleBarrierSymmetric"}
	\caption{Triple Barrier Labeling (Symmetric barriers)}
	\label{fig:tripleBarrierSymmetric}
\end{figure}

\subsection{Adaptation when the side is known}
\label{sec:tripleBarrierSide}
Noting the side of the trade as $\hat{y}^{\text{M1}}_{i} \in 
\{1,-1\}$, that is, either long or short. Accordingly, the method can 
be modified to include these changes:

\begin{equation}
	r_i = \left( \left[ (1 - \text{tc})^2 \cdot 
	\frac{p_{t_{i,1}}}{p_{t_{i,0}}} \right] - 1 \right) \cdot 
	\hat{y}^{\text{M1}}_{i}
\end{equation}

Combining everything, the observations are labeled as:
\begin{equation}
	\hat{y}^{\text{M2}}_i =
    \begin{cases}
      1 & \text{if}\ r_i > 0 \\
      0 & \text{otherwise}
    \end{cases}
\end{equation}

Apart from that, the horizontal barriers should be changed since it 
has been predicted that the stock will go up/down (long/short):

\begin{equation}
	\delta_{+, t_{i,0}} =
    \begin{cases}
      \text{pt} \cdot \sigma_{t_{i,0}-1} & \text{if}\ 
      \hat{y}^{\text{M1}}_{i} = 1 \\
      \min(0.005,\ 0.5 \cdot \text{pt} \cdot \sigma_{t_{i,0}-1}) & 
      \text{otherwise}
    \end{cases}
\end{equation}

\begin{equation}
	\delta_{-, t_{i,0}} =
    \begin{cases}
      \text{sl} \cdot \sigma_{t_{i,0}-1} & \text{if}\ 
      \hat{y}^{\text{M1}}_{i} = -1 \\
      \min(0.005,\ 0.5 \cdot \text{sl} \cdot \sigma_{t_{i,0}-1}) & 	
      \text{otherwise}
    \end{cases}
\end{equation}

Therefore, the price will oscillate between $[ p_{t_{i,0}} - 
\delta_{-, t_{i,0}}, \ p_{t_{i,0}} + \delta_{+, t_{i,0}} ]$. Also, 
note that $t_{i,0} - 1$ represents the last trading day before 
$t_{i,0}$ because to compute the barriers of dat $t_{i,0}$ only 
information \textbf{up to that day} should be used.\\

In figure \ref{fig:tripleBarrierSide} an example can be seen which 
presents a side $= -1$ (short). As the prediction says the price 
should go down, the upper horizontal barrier has been lowered. In this 
case, since the price has hit the upper horizontal barrier and the 
side was $-1$, then the observation will be \textbf{labeled 0}.

\begin{figure}[htbp]
	\centering
	\includegraphics[width=.75\textwidth]
	{"\homeCOne/img/tripleBarrierSide"}
	\caption{Triple Barrier Labeling - Side $= -1$ (Short)}
	\label{fig:tripleBarrierSide}
\end{figure}

\section{Models}
\label{sec:models}
Bet-sizing is a common problem within the practitioner scene. That is, 
investors often know the side of the position they want to open 
(long/short or pass) but the size is unknown to them. Hence, an 
exogenous model that could base its predictions on some features, and 
the side prediction, comes into play.\\

The structure will be the following:

\begin{itemize}
	\item \textbf{Primary Model:} A set of features is used to give a 
	prediction on the side of the position. It can be based on 
	fundamental analysis, ML, technical analysis, etc.
	
	\item \textbf{Secondary Model:} This model uses another set of 
	features that contains the prediction of the primary model. It 
	aims to give a prediction on the size of the position.
	
	\item \textbf{Meta Model:} It decides the side (primary model) and 
	the size (secondary model). It is merely a combination of both 
	models.
\end{itemize}

The following example shows how the models work:

\begin{gather*}
	f_{primary}(\textbf{$X_{\text{M1}}$}) = 0.89\\
	f_{secondary}(\{0.89, \textbf{$X_{\text{M2}}$}\}) = 0.33
\end{gather*}

\begin{table}[htbp]
\caption{Example Metalabeling}
\label{table:exMetalabeling}
\centering
\begin{tabular}{|C{3cm}|C{3cm}|C{3cm}| }
	\hline
	Model & Threshold & Prediction\\
	\hline
	Primary & 0.63 & 0.89\\
	\hline
	Secondary & 0.41 & 0.33\\
	\hline
\end{tabular}
\end{table}

Considering that $Im(f_{primary}) \subset [0, 1]$ (i.e., no shorting) 
and $Im(f_{secondary}) \subset [0, 1]$ then, one can conclude from the 
results in table \ref{table:exMetalabeling} that:

\begin{gather*}
		0.89 > 0.63 = \text{threshold}_{\text{M1}} \Rightarrow 
		\text{Long position}\\
		0.33 \leq 0.41 =\text{threshold}_{\text{M2}} \Rightarrow 
		\text{Not enough confidence to open the position}
\end{gather*}

The primary model hinted that a long position should be opened. 
However, the secondary model evaluated the choice of the primary model 
and decided not to open the position, since it did not meet the 
principal criteria: exceeding the threshold. Note that the threshold 
column in table \ref{table:exMetalabeling} will be determined in a way 
that maximizes a carefully chosen performance metric.\\

Summing all up:

\begin{itemize}
	\item Develop a primary model (M1) that determines the side 
	of a trade
	\item \textbf{Whenever M1 says that one should open a position}, 
	determine whether M1 gave a correct output or it was wrong 
	(\textbf{meta-labeling - labeling a model})
	\item With the output of M1, a set of features (they can be the 
	same) and the previous labels develop a secondary model (M2)
	\item If M1 tells you to trade and M2's prediction surpasses the 
	threshold of M2, open a position with the side given from M1.
\end{itemize}

\section{Binary classification problem}
Now that the way observations will be labeled has been explained, and 
the models involved have been presented, it is relevant to address the 
binary classification problem that this chapter attempts to solve.\\

The events that one wants to predict are the entries to profitable 
investment opportunities, i.e.,:

\begin{itemize}
	\item 1 = \textbf{Open a position}
	\item 0 = \textbf{Do not open a position}
\end{itemize}

% Last remark

Having laid out the objective of the problem, the following variables 
will be defined:

\begin{itemize}
	\item \textbf{True Positive} (TP) 
	$\hat{y}^{\text{M2}}_i = 1 = y^{\text{M2}}_i$
		
	\item \textbf{False Positive} (FP) 
	$\hat{y}^{\text{M2}}_i = 1 \neq y^{\text{M2}}_i$
	
	\item \textbf{True Negative} (TN) 
	$\hat{y}^{\text{M2}}_i = 0 = y^{\text{M2}}_i$
	
	\item \textbf{False Negative} (FN) 
	$\hat{y}^{\text{M2}}_i = 0 \neq y^{\text{M2}}_i$
	
	\item \textbf{Recall} $ = \frac{\text{TP}}{\text{TP} + \text{FN}}$
	
	\item \textbf{Precision} 
	$ = \frac{\text{TP}}{\text{TP} + \text{FP}}$
	
	\item \textbf{F1 - Score} 
	$ = \frac{2}{\text{recall}^{-1} + \text{precision}^{-1}}$
\end{itemize}

Where $\hat{y}^{\text{M2}}$ are the predictions and $y^{\text{M2}}$ 
are the true labels of M2.\\

The idea behind meta-labeling lays in dividing the classification task 
into two parts. First, a model will act as an initial filter that has 
high recall, i.e., it catches most of the profitable opportunities, 
some being false positives. The second model, whenever the first has 
predicted that there is a profitable opportunity, will 
``double-check'' the supposed positive and will decide whether it is 
a positive or a negative. By doing that, the overall precision will 
improve, getting better F1-scores (since recall was initially high).

\section{Primary Model}
\label{sec:primaryModel}
The main role of the primary model is to find investment opportunities 
and output the side (long/short). In a sense, it could be thought as a 
meta model where the secondary model always tell you to open a 
position.

\subsection{MA based Primary Model}
\label{sec:MAPrimaryModel}
This primary model uses a Moving Average (MA) crossover strategy with 
a 20 day look-back period to output the side of the investment. The 
day of entry will be determined with the CUSUM filter. It can all be 
summed up in the following equations:

\begin{equation}
\label{eqn:MA}
	\text{MA}_t = \frac{\sum_{i = 0}^{19} p_{t-i}}{20}
\end{equation}

Whit the MA computed, the predicted side of the position follows:

\begin{equation}
\label{eqn:MAcrossover}
	\hat{y}_{t+1}^{\text{M1}} =
	\begin{cases}
		1  & \text{if}\ \text{MA}_t < p_t\\
		-1 & \text{if}\ \text{MA}_t \geq p_t
	\end{cases}
\end{equation}

Note that the side generated is of the following trading day ($t+1$) 
because the information ($\text{MA}_t$) is received at the end of day 
$t$ and one can invest with those insights the next trading day.\\

On the other hand, the true labels are:

\begin{equation}
	y_{i}^{\text{M1}} = 
	\begin{cases}
	\hfill 1 & \text{if}\ 
	\tilde{r}_i = \frac{p_{t_{i,1}}}{p_{t_{i,0}}} -1 > 0\\
	\hfill -1 & \text{otherwise}	
	\end{cases}
\end{equation}

The entry points will be determined by structural breaks identified by 
the symmetric CUSUM (cumulative sum) filter, defined by:

\begin{gather*}
	\text{CUSUM}_{+,\ t} = \max \left( 0,\ \text{CUSUM}_{+,\ t-1} + 
	\log \left( \frac{p_t}{p_{t-1}} \right) \right)\\	
	\text{CUSUM}_{-,\ t} = \min \left( 0,\ \text{CUSUM}_{-,\ t-1} + 
	\log \left( \frac{p_t}{p_{t-1}} \right) \right)
\end{gather*}

With the boundary conditions $\text{CUSUM}_{+,\ 0} = 
\text{CUSUM}_{-,\ 0} = 0$.\\

The filters \textbf{are reset to 0 whenever}
$\text{CUSUM}_{+,\ t} > \hat{\sigma}_t$ or $\text{CUSUM}_{-,\ t} < 
-\hat{\sigma}_t$, where $\hat{\sigma}_t$ is the volatility of 
logarithmic returns using a 20 day window. Whenever this reset 
happens, it will be said that \textbf{the filters have been 
activated.}\\

Hence at this point, at day $t+1$ one should enter if either of the 
CUSUM filters are activated at day $t$ and there are no positions 
open. The side is computed using the formula shown in equation 
\ref{eqn:MAcrossover}.\\

Finally, to exit the position, one needs to compute the vertical and 
horizontal barriers as it was explained in subsection 
\ref{sec:tripleBarrierSide}. On the occasion that the price time 
series moves past a barrier, one should  proceed to close the 
position.

\subsection{ML based Primary Model}
\label{sec:MLPrimaryModel}
This model is slightly different that the one based on MA. It aims to 
train a neural network with a hidden layer (20 fully connected units 
- Leaky ReLU) and an output layer (Sigmoid). It will be in charge of 
predicting the side (1 = long, -1 = short).\\

Firstly, let's present the features that the neural network will use. 
Note that the features of day $\textbf{t+1}$ use information 
\textbf{only} from previous days, not the current day.

\begin{itemize}
	\item $\text{F1}_{t+1} \equiv \log \left( \frac{\text{MA}_t}{p_t} 
	\right)$
	
	\item $\text{F2}_{t+1} \equiv \frac{\text{CUSUM}_{+,\ t}}{
	\sigma_t}$
	
	\item $\text{F3}_{t+1} \equiv \frac{\text{CUSUM}_{-,\ t}}
	{\sigma_t}$
	
	\item $\text{F4}_{t+1} \equiv \textbf{Reset}_{\text{CUSUM}_{+},t}$ 
	: Binary variable that indicates if the positive CUSUM filter 	
	became activated at time $t$.
	
	\item $\text{F5}_{t+1} \equiv \textbf{Reset}_{\text{CUSUM}_{-,t}}$
	: Binary variable that indicates if the negative CUSUM filter 	
	became activated at time $t$.
	
	\item $\text{F6}_{t+1} \equiv \text{EWMSD}_t$ 
	: Exponentially weighted moving standard deviation of linear 
	returns.
\end{itemize}

Where $\sigma_t^2 = \frac{\sum_{i=0}^{19}(r_{t-i} - 
\bar{r}_{t-19, t})^2}{20 - 1}$.\\

In this model one does not depend on the CUSUM filter to determine the 
date to open a position. The neural network, with the  features given, 
will decide the side and the position will be opened. That being said, 
with the intention of matching features and labels, the latter have 
been defined as:

\begin{equation}
	y_{i}^{\text{M1}} = 
	\begin{cases}
	\hfill 1 & \text{if}\ \tilde{r}_i > 0\\
	\hfill -1 & \text{otherwise}	
	\end{cases}
\end{equation}

Where $\tilde{r}_i = \frac{p_{t_{i,1}}}{p_{t_{i,0}}} -1 $ is the 
realized return using the triple barrier method, $t_{i,0}$ the start 
day and $t_{i,1}$ the day one of the barriers is hit. Also, remember 
that since side is unknown, the horizontal barriers will be defined 
with $\delta_{+, t_{i,0}} = \delta_{-, t_{i,0}} = \sigma_{t_{i,0}-1}$.

\section{Secondary Model}
\label{sec:secondaryModel}
As it was explained in section \ref{sec:models} the idea behind a 
secondary model is to train a model capable of learning how to use the 
primary model, deciding when one should/should not enter. Being more 
specific, it will filter the positives predicted from the primary 
model.\\

Similarly to what it was done in section \ref{sec:primaryModel} the 
models will be adapted to the 2 primary models defined.

\subsection{MA based Secondary Model}
\label{sec:MASecondaryModel}
This model will use a random forest trained with the following
\textbf{labels:}

\begin{equation*}
	y^{\text{M2}}_i =
	\begin{cases}
	1 & \text{if}\ y^{\text{M1}}_i = \hat{y}^{\text{M1}}_i\\
	0 & \text{otherwise}
	\end{cases}
\end{equation*}

Note that this definition matches the one in subsection 
\ref{sec:tripleBarrierSide} since:

\begin{equation*}
	y^{\text{M1}}_i = \hat{y}^{\text{M1}}_i \Rightarrow 
	r_i	= \left( \frac{p_{t_{i,1}}}{p_{t_{i,0}}} - 1 \right) \cdot 
	\hat{y}^{\text{M1}}_i > 0
\end{equation*}

Additionally, the following features for day $\boldsymbol{t_{i,0}}$ 
will be used:

\begin{itemize}	
	\item $\log \left( \frac{\text{MA}_{(t_{i,0}-1)}}{p_{(t_{i,0}-1)}} 
	\right)$
	
	\item $r_{(t_{i,0}-1)}$: return of the previous day.
	
	\item $\underline{r_{(t_{i,0}-1)}} \equiv \left( 
	\prod_{k=0}^{4} (1 + r_{t_{i,0}-k-1}) 
	\right) - 1$: Cumulative return of a 5 day window.
	
	\item $\hat{y}_{i}^{\text{M1}}$

	\item $\text{RSI}_{(t_{i,0}-1),\ 9},\ 
	\text{RSI}_{(t_{i,0}-1),\ 14},\ 
	\text{RSI}_{(t_{i,0}-1),\ 25}$:
	Relative Strength Indicator (RSI) of linear returns using a 9, 14 
	and 25 days window respectively.
	
	\item $\tilde{\sigma}_{(t_{i,0}-1)}$: Exponentially weighted 
	volatility using a 21 day ($\approx$ 1 trading month) window.
	
	\item $\sigma_{(t_{i,0}-1),\ 9},\ \sigma_{(t_{i,0}-1),\ 14},\ 
	\sigma_{(t_{i,0}),\ 25}$: Volatility of linear returns using a 9, 
	14 and 25 days window respectively.
	
	\item $R_{(t_{i,0}-1),\ 1},\ R_{(t_{i,0}-1),\ 5}$: 
	Auto-correlation function of returns with 1 and 5 days lag 
	respectively.
\end{itemize}

The meta model (primary + secondary), together with the threshold 
$\text{thr}_{\text{M1}}$, will work using the following cases:

\begin{enumerate}
	\item Check if a position is currently open
	
	\item If no position is open then compute the side 
	($\hat{y}_i^{\text{M1}}$) and predict the size 
	($\hat{y}_i^{\text{M2}}$) using the primary and secondary model 
	respectively.
	
	\item 
	Go long if: 
		\begin{itemize}
			\item $\hat{y}_i^{\text{M1}} = 1$ and 
			$\hat{y}^{\text{M2}} > \text{thr}_{\text{M2}}$ - The 
			primary model was right about going long.
		\end{itemize}
	
	Go short if:
		\begin{itemize}
			\item $\hat{y}_i^{\text{M1}} = -1$ and 
			$\hat{y}^{\text{M2}} > \text{thr}_{\text{M2}}$ - The 
			primary model was right about going short.
		\end{itemize}

	Otherwise, the secondary model infers that there is not enough 
	confidence to open a position and one should return to point 1 
	using data from the	next possible entry point.
	
	\item Open a position and exit using the triple barrier method.
	
	\item Return to point 1 using data from the next possible entry 
	point.
\end{enumerate}

\subsection{ML based Secondary Model}
\label{sec:MLSecondaryModel}
This model will use a neural network with a hidden layer (25 fully 
connected units - Leaky ReLU) and an output layer (Sigmoid). The 
labeling technique is the same as the one found in 
\ref{sec:MASecondaryModel}:

\begin{equation*}
	y^{\text{M2}}_i =
	\begin{cases}
	1 & \text{if}\ y^{\text{M1}}_i = \hat{y}^{\text{M1}}_i\\
	0 & \text{otherwise}
	\end{cases}
\end{equation*}
The features for day $\boldsymbol{t_{i,0}}$ are the same as the ones 
found in the MA based Secondary Model (\ref{sec:MASecondaryModel}): \\

$\hat{y}_i^{\text{M1}},\
\log \left( \frac{\text{MA}_{(t_{i,0}-1)}}{p_{(t_{i,0}-1)}} \right),\
r_{(t_{i,0}-1)},\
\underline{r_{(t_{i,0}-1)}},\
\hat{y}_i^{\text{M1}},\
\text{RSI}_{(t_{i,0}-1),\ 9},\
\text{RSI}_{(t_{i,0}-1),\ 14},\
\text{RSI}_{(t_{i,0}-1),\ 25},\
\tilde{\sigma}_{(t_{i,0}-1)},\\
\sigma_{(t_{i,0}-1),\ 9},\
\sigma_{(t_{i,0}-1),\ 14},\
\sigma_{(t_{i,0}-1),\ 25},\
R_{(t_{i,0}-1),\ 1} \text{ and } R_{(t_{i,0}-1),\ 5}$


\vspace{.5cm}

The meta model (primary + secondary) will follow the same procedure 
from subsection \ref{sec:MLSecondaryModel} to open/close a position.

\section{Hyper-parameter tuning via cross-validation}
\subsection{Cross-Validation MA based Models}
As it has been seen in subsections \ref{sec:MAPrimaryModel}, 
\ref{sec:MLPrimaryModel}, \ref{sec:MASecondaryModel} and 
\ref{sec:MLSecondaryModel}, thresholds are important parameters when 
deciding whether to open a position or not. Consequently, attention 
should be paid so as to achieve the best performance possible. This is 
where the Sharpe Ratio comes into play.\\

Suppose one has obtained a daily time series ($r_t$) of returns from a 
model. Then one can define the annualized Sharpe Ratio:

\begin{equation}
	\text{SR} = \frac{\text{mean}(\hat{r}_t)}{\text{sd}(\hat{r}_t)} 
	\cdot \sqrt{\frac{\# \text{oportunities}}{\# \text{years}}}
\end{equation} 

Where $\hat{r}_t$ is the time series without the zero returns (days 
where the model has decided not to enter), $\#\text{opportunities}$ is 
the number of observations of the time series $\hat{r}_t$ and 
$\# \text{years}$ are the number of years elapsed from the first to 
the last observation of the time series $\hat{r}_t$.\\

It is important to point out that a Buy \& Hold strategy (buying the 
portfolio and only re balancing periodically) presents the following 
properties: 

\begin{enumerate}
	\item $\hat{r}_t = r_t$ because the strategy keeps always a 
	position open. Hence, there are 0 days with zero returns.
	
	\item $\frac{\# \text{opportunities}}{\# \text{years}} = 252$ 
	(Trading days in a year) for the same reason as in point 1.	
\end{enumerate}

Having defined all these variables, the idea behind having a separate 
data set that has not been explicitly trained on, is to determine the 
threshold that performs best. More accurately, the threshold that 
obtains the best Sharpe Ratio, given that it has enough observations 
to assure that the Sharpe Ratio is not misleading.\\

The criteria that has been used to determine whether there are enough 
observations is the following:

\begin{itemize}
	\item For every threshold compute the number of times the model 
	decides to enter, which will be called number of discrete returns. 
	Note that it is different than the number of days the model has 
	kept a position open.
	
	\item Compute the median.
	
	\item If the number of discrete returns surpasses half the median,
	then it is labeled ``Enough observations''.
\end{itemize}

The idea behind this filter is to avoid over fitting. By choosing a 
threshold that has a low number of discrete returns then it is likely 
that it will not perform as good out-of-sample because there are few 
observations.\\

Since the primary model is based on MA crossovers, which are 
deterministic, there are not hyper parameters to tune. However, the 
CUSUM filter determined that a position should be opened when 
$\text{CUSUM}_{+,\ t} > \hat{\sigma}_t$ or 
$\text{CUSUM}_{-,\ t} < -\hat{\sigma}_t$. This can be adapted by 
scaling the volatility using $\text{thr}_{\text{CUSUM}}$, i.e., 
opening a position (and resetting) whenever:

\begin{itemize}
	\item $\text{CUSUM}_{+,\ t} > \text{thr}_{\text{CUSUM}} \cdot
	\hat{\sigma}_t$
	\item $\text{CUSUM}_{-,\ t} < -\text{thr}_{\text{CUSUM}} \cdot
	\hat{\sigma}_t$
\end{itemize}


As it can be seen in table \ref{tab:CV_MA_M1}, the results are far 
from satisfactory. Negative Sharpe Ratios are misleading since a small 
standard deviation can throw things off when returns are close to 
0 but negative.\\

\begin{table}[htbp]
\caption{Results cross-validation Primary Model (MA)}
\label{tab:CV_MA_M1}
\centering
\begin{tabular}{ |C{3cm}|C{3cm}|C{3cm}| }
	\hline
	$\text{thr}_{\text{CUSUM}}$ & Final wealth & SR Primary 
	Model (Validation)\\
	\hline
	0.25 & 0.6533 & -1.3585\\ 
	0.60 & 0.6449 & -1.4021\\ 
	0.95 & 0.6566 & -1.4137\\  % Meta model va muy bien (SR 0.4)
	1.30 & 0.6290 & -1.6683\\  % Meta model va mal (SR -0.24)
	1.65 & 0.6758 & -1.5189\\ % Meta (SR -0.27)
	2.00 & 0.7419 & -1.1885\\ 
	\hline
\end{tabular}
\end{table}

That is why it has been decided to determine 
$\text{thr}_{\text{CUSUM}}$ indirectly. This will be done by computing 
the best Sharpe Ratio (using the meta model) for every 
$\text{thr}_{\text{CUSUM}}$. Lastly, the threshold that enables the 
meta model to achieve the highest SR will be chosen. Table 
\ref{tab:results_SR_MA_M2} summarizes the results and figures 
\ref{fig:thr1} through \ref{fig:thr6} get into detail about the SR 
obtained for every threshold.\\

From table \ref{tab:results_SR_MA_M2} one can conclude that 
$\text{thr}_{\text{CUSUM}} = 0.25$ performs best, since it achieves a 
SR of 0.4971. Hence, this threshold will be fixed from now on.

\begin{table}[htbp]
\caption{Results cross-validation Meta Model (MA)}
\label{tab:results_SR_MA_M2}
\centering
\begin{tabular}{ |C{3.5cm}|C{3.5cm}| }
	\hline
	$\text{thr}_{\text{CUSUM}}$ & SR Meta Model (Validation)\\
	\hline
	0.25 &  0.4971\\
	0.60 & -0.1333\\
	0.95 &  0.4230\\
	1.30 &  0.0051\\
	1.65 &  0.2348\\
	2.00 & -0.4674\\
	\hline
\end{tabular}
\end{table}

\begin{figure}[htbp]
\centering
\begin{minipage}{.5\textwidth}
  \centering
  \includegraphics[scale=.25]{"\homeCOne/img/CV_MA_M2_thr1"}
  \caption{SR MM CV ($\text{thr}_{\text{CUSUM}} = 0.25$)}
  \label{fig:thr1}
\end{minipage}%
\begin{minipage}{.5\textwidth}
  \centering
  \includegraphics[scale=.25]{"\homeCOne/img/CV_MA_M2_thr2"}
  \caption{SR MM CV ($\text{thr}_{\text{CUSUM}} = 0.60$)}
  \label{fig:thr2}
\end{minipage}
\end{figure}

\begin{figure}[htbp]
\centering
\begin{minipage}{.5\textwidth}
  \centering
  \includegraphics[scale=.25]{"\homeCOne/img/CV_MA_M2_thr3"}
  \caption{SR MM CV ($\text{thr}_{\text{CUSUM}} = 0.95$)}
  \label{fig:thr3}
\end{minipage}%
\begin{minipage}{.5\textwidth}
  \centering
  \includegraphics[scale=.25]{"\homeCOne/img/CV_MA_M2_thr4"}
  \caption{SR MM CV ($\text{thr}_{\text{CUSUM}} = 1.30$)}
  \label{fig:thr4}
\end{minipage}
\end{figure}

\begin{figure}[htbp]
\centering
\begin{minipage}{.5\textwidth}
  \centering
  \includegraphics[scale=.25]{"\homeCOne/img/CV_MA_M2_thr5"}
  \caption{SR MM CV ($\text{thr}_{\text{CUSUM}} = 1.65$)}
  \label{fig:thr5}
\end{minipage}%
\begin{minipage}{.5\textwidth}
  \centering
  \includegraphics[scale=.25]{"\homeCOne/img/CV_MA_M2_thr6"}
  \caption{SR MM CV ($\text{thr}_{\text{CUSUM}} = 2.00$)}
  \label{fig:thr6}
\end{minipage}
\end{figure}

\vspace{5cm}

\subsection{Cross-Validation ML based Models}
The purpose of cross-validation in the ML based models has been 
different than the ones based in MA. The validation data set has been 
used as an independent stopping indicator. That is, the neural 
networks have been trained on the train data set and whenever the loss 
function value in the validation data set increases, while the train 
one decreases, the algorithm stops. This way, over fitting is avoided 
since the neural network was stopped before it became too good a 
predictor in a certain data set, without delivering the same results 
in unseen data sets.\\

That said, the threshold for both the primary and secondary model has 
not been tuned. So a $\hat{y}_i$ will be given a positive value 
whenever it surpasses 0.5, and negative otherwise.

\section{Results}
In subsections \ref{sec:ResultsMA} and \ref{sec:ResultsML} a table and 
some graphs will be shown with the purpose of assessing the 
out-of-sample performance (Sharpe Ratio \& Drawdown) of the models 
considered, which are the following:

\begin{itemize}
	\item \textbf{Buy \& Hold:} Buying the portfolio at the start and 
	keeping the position until the end of the period.
	
	\item \textbf{Primary Model:} Similar as the previous model but it 
	applies the corresponding side (long/short).
	
	\item \textbf{Meta Model:} Combination of the primary model and 
	the secondary model. The latter is used to filter the trades 
	from the former.
\end{itemize}

\subsection{MA based Models}
\label{sec:ResultsMA}
Since a huge emphasis has been put to select the threshold that 
performs best, SR-wise, in the validation set (in-sample), the results 
of the MA-based model in the validation data set are shown in table 
\ref{tab:ResultsCV}.\\

As it is seen in table \ref{tab:ResultsCV}, the meta model does not 
even achieve satisfactory results (beating the SR of B\&H) in the 
validation model, so one should not expect to see positive results in 
the Test Dataset. Indeed, in table \ref{tab:ResultsTst}, the results 
in the test data set show how the meta model is unable to surpass the 
SR of the B\&H.\\

Besides that, the precision of the primary model is so low (0.51 -
validation, 0.58 - test) that the meta model has needed to lower the 
recall considerably in order to achieve better precision results. That 
leads to a decrease in F1-Score in both data sets.\\

For more information, one can refer to figures 
\ref{fig:Plots_MA_Validation} through \ref{fig:Drawdown_MA_Test}, 
where the prices and drawdown of the models have been plotted.

\begin{table}[htbp]
\caption{Results in validation data set (MA)}
\label{tab:ResultsCV}
\centering
\begin{tabular}{ |C{3cm}|C{3cm}|C{2cm}|C{2cm}|C{2cm}|C{2cm}|}
	\hline
	Model 			& Max. Drawdown	& SR			& Precision & Recall 
	& F1-Score\\
	\hline
	Buy \& Hold		&  9.17\% 		&  0.9479	& -		 & - 
	& -\\  
	Primary Model 	& 41.36\% 		& -1.3585	& 0.5100 & 1 
	& 0.6757\\ 
	Meta Model 		&  4.97\% 		&  0.4504	& 0.6444 & 0.1657
	& 0.2636\\ 
	\hline
\end{tabular}
\end{table}

\begin{table}[htbp]
\caption{Results in test data set (MA)}
\label{tab:ResultsTst}
\centering
\begin{tabular}{ |C{3cm}|C{3cm}|C{2cm}|C{2cm}|C{2cm}|C{2cm}|}
	\hline
	Model 			& Max. Drawdown & SR			& Precision & Recall
	& F1-Score\\
	\hline
	Buy \& Hold		& 15.20\% 		&  0.9682	& - 		 & - 
	& -\\ 
	Primary Model 	& 34.41\% 		& -0.7467	& 0.5758 & 1 
	& 0.7308\\ 
	Meta Model 		&  5.02\% 		&  0.4887	& 0.6769 & 0.1781
	& 0.2821\\ 
	\hline
\end{tabular}
\end{table}

\begin{figure}[htbp]
	\centering
	\includegraphics[scale=.5]{"\homeCOne/img/Plots_MA_Validation"}
	\caption{Returns of MA Models in the Validation Dataset}
	\label{fig:Plots_MA_Validation}
\end{figure}

\begin{figure}[htbp]
	\centering
	\includegraphics[scale=.5]{"\homeCOne/img/Drawdown_MA_Validation"}
	\caption{Drawdown of MA Models in the Validation Dataset}
	\label{fig:Drawdown_MA_Validation}	
\end{figure}

\begin{figure}[htbp]
	\centering
	\includegraphics[scale=.5]{"\homeCOne/img/Plots_MA_Test"}
	\caption{Returns of MA Models in the Test Dataset}
	\label{fig:Plots_MA_Test}
\end{figure}

\begin{figure}[htbp]
	\centering
	\includegraphics[scale=.5]{"\homeCOne/img/Drawdown_MA_Test"}
	\caption{Drawdown of MA Models in the Test Dataset}
	\label{fig:Drawdown_MA_Test}
\end{figure}

\vspace{100cm}


\subsection{ML based Models}
\label{sec:ResultsML}
Table \ref{tab:ResultsTstML} shows the results of the different models 
based on ML in the Validation and Test Dataset. Figures 
\ref{fig:Plots_ML_Test} and \ref{fig:Drawdown_ML_Test} show the prices 
and drawdown for the different models in the test data set.\\

The immediate thing to point out is that the performance between the 
primary model and the meta model is indistinguishable. That means that 
the secondary model has predicted that the primary model was right all 
of the times. That is attributed to the lack of predictive power of 
the features of the secondary model. Consequently, the secondary 
model has decided to ignore the features and always predict a 1 since 
it is the majority class (it will result in the best naive metrics). 
Apart from that, in the primary model a similar thing happens since it 
always predicts to go long.\\

All of this is reflected in the metrics, which fail to change from 
model to model:

\begin{table}[htbp]
\caption{Results in Test Dataset (ML)}
\label{tab:ResultsTstML}
\centering
\begin{tabular}{ |C{3cm}|C{3cm}|C{1.8cm}|C{1.8cm}|C{1.8cm}|C{1.8cm}| }
	\hline
	Model 			& Max. Drawdown 	& SR			& Precision & Recall 
	& F1-Score\\
	\hline
	Buy \& Hold		& 15.20\% 		&  0.9682	& - 		 & - 
	& -\\ 
	Primary Model 	& 16.42\% 		&  0.6630	& 0.6757 & 1 
	& 0.8065\\ 
	Meta Model 		& 16.42\% 		&  0.6630	& 0.6758 & 0.9969 
	& 0.8055\\ 
	\hline
\end{tabular}
\end{table}

\begin{figure}[htbp]
	\centering
	\includegraphics[scale=.5]{"\homeCOne/img/Plots_ML_Test"}
	\caption{Prices of ML Models in the Test Dataset}
	\label{fig:Plots_ML_Test}
\end{figure}

\begin{figure}[htbp]
	\centering
	\includegraphics[scale=.5]{"\homeCOne/img/Drawdown_ML_Test"}	
	\caption{Drawdown of ML Models in the Test Dataset}	
	\label{fig:Drawdown_ML_Test}
\end{figure}

\vspace{100cm}

\section{Coin flip correction}
\label{coinFlip} 
As it has been seen in the previous sections, the performance of the 
primary model has been sub optimal. In an effort to remove this 
condition from this work, a strategy has been devised towards 
achieving better (but artificial) primary models.\\

The idea lays in the concept of a coin flip, $S \sim Be(p)$ 
($p = \text{Pr}(S = 1)$). Suppose that every observation is linked to 
a random variable $S_i \sim Be(p)$. Then, one can define a new 
feature:

\begin{gather*}
	F_i = (1 - S_i) \cdot y_i - S_i \cdot y_i = 
	(1 - 2 \cdot S_i) \cdot y_i\\
	\mathbb{E}[F_i\ |\ y_i] = (1 - 2p) \cdot y_i
\end{gather*}

where $y_i \in \{-1,\ 1\}$ is the label of the observation in the 
primary model.\\

Interpreting $p$ as the probability of swapping a feature, the model 
that defines $S_i \sim Be(0)$ will not swap features and the primary 
model will be the ``perfect classifier''. On the other hand, if 
$S_i \sim Be(0.5)$ then $\mathbb{E}[F_i\ |\ y_i] = 0$ and the 
feature is useless since the feature will alternate (in a random way) 
between $-1$ and $1$.\\

These artificial models are based on the ML primary model. That is, 
apart from the \textit{coin flip} feature, it conserves the same 
features. The probabilities that define the r.v. $S_i$ are:

\begin{equation*}
	\textbf{p} = [0.00, 0.05, 0.10, 0.15, 0.20, 0.25, 0.30, 0.35, 
	0.40, 0.45, 	0.50]
\end{equation*}

\subsection{Results}

In figures \ref{fig:coinFlipPlots1}, \ref{fig:coinFlipPlots2} and 
\ref{fig:coinFlipPlots3} the P\&L plots of the artificial Primary and 
Meta Model for the Test Dataset are shown.

\begin{figure}[hbtp]
	\centering
	\includegraphics[scale=.5]{"\homeCOne/img/coinFlip/coinFlipPL1"}
	\caption{Out-of-sample P\&L for 
	$p \in \{ 0.00, 0.05, 0.10, 0.15\}$}
	\label{fig:coinFlipPlots1}
\end{figure}

\begin{figure}[htbp]
	\centering
	\includegraphics[scale=.5]{"\homeCOne/img/coinFlip/coinFlipPL2"}
	\caption{Out-of-sample P\&L for 
	$p \in \{ 0.20, 0.25, 0.30, 0.35\}$}
	\label{fig:coinFlipPlots2}
\end{figure}

\begin{figure}[htbp]
	\centering
	\includegraphics[scale=.5]{"\homeCOne/img/coinFlip/coinFlipPL3"}
	\caption{Out-of-sample P\&L for $p \in \{ 0.40, 0.45, 0.50 \}$}
	\label{fig:coinFlipPlots3}
\end{figure}

\begin{figure}[htbp]
	\centering
	\includegraphics[scale=.5]
	{"\homeCOne/img/coinFlip/coinFlipModelsSrTest"}
	\caption{Sharpe Ratio of different coin flip models 
	(\textit{Out-of-sample})}
	\label{fig:coinFlipModelsSrTest}
\end{figure}

\begin{figure}[htbp]
	\centering
	\includegraphics[scale=.5]{"\homeCOne/img/coinFlip/coinFlipRecall"}
	\caption{Recall of Coin Flip Models (\textit{Out-of-sample})}
	\label{fig:coinFlipRecallTest}
\end{figure}

\begin{figure}[htbp]
	\centering
	\includegraphics[scale=.5]{"\homeCOne/img/coinFlip/coinFlipPrecision"}
	\caption{Precision of Coin Flip Models (\textit{Out-of-sample})}
	\label{fig:coinFlipPrecisionTest}
\end{figure}

\begin{figure}[htbp]
	\centering
	\includegraphics[scale=.5]
	{"\homeCOne/img/coinFlip/coinFlipF1Score"}
	\caption{F1 Score of Coin Flip Models (\textit{Out-of-sample})}
	\label{fig:coinFlipF1ScoreTest}
\end{figure}

\vspace{100cm}

\section{Conclusions}
This chapter has explored the novel labeling technique meta-labeling. 
Beginning from the toy project, it was obvious that the main benefit 
of meta-labeling, improving the F1-Score, was hard to capture. In 
fact, the secondary model needed low levels of noise in order to work, 
besides having a primary model that was not near perfect (i.e., it 
made mistakes to could be corrected).\\

The conditions were obviously not met in the MA and ML based models. 
In the former, the primary model predictions were near random with 
precision of nearly 50\%. That created a really bad model which was 
hard for the secondary model to correct. Even though it achieved some 
success in improving the SR, it was unable to beat a B\&H strategy. 
Apart from that, the ML based models did not have enough data with 
predictive power, so the results of meta-labeling are inconclusive. In 
fact, one would dare to say that the models did not ``learn'' anything 
which is common in systems where the models can not distinguish the 
signal from the noise. Note that, this did not happen in the MA based 
model because even tough it was almost random, it did something. In 
contrast the ML M1 did not have anything to base its predictions on, 
so it decided to always go long, since it would achieve the ``best'' 
performance for a naive classifier.\\

However, not everything is unfavorable. By creating better artificial 
primary models (coin-flip method), the hypothesis in the toy project 
could be evaluated in a more controlled environment. It can be seen 
that the SR of the meta model significantly surpasses the one of the 
primary model whenever the probability of the coin flip is between 0.1 
and 0.3. That fits with what was concluded in the toy project: M1 not 
near perfect and low noise-to-signal ratio.\\

All in all, meta-labeling applied to financial data is a tricky tool 
to use. First of all, a good primary model is required, which by 
itself is difficult to develop. Furthermore, the features of the 
secondary model need to have good predicting power so as to detect the 
signal behind the noise, which in financial data is challenging as 
well .

\begin{table}[htbp]
\caption{Tickers of the GMVP portfolio}
\label{table:tickers}
\centering
\begin{tabular}{ |C{2cm}|C{2cm}|C{2cm}|C{2cm}|C{2cm}|C{2cm}|  }
	\hline
	MMM & ABT & ABBV & ABMD & ACN & ATVI\\
	ADBE & AMD & AAP & AES & AFL & A\\
	APD & AKAM & ALK & ALB & ARE & ALXN\\
	ALGN & ALLE & LNT & ALL & GOOGL & GOOG\\
	MO & AMZN & AMCR & AEE & AAL & AEP\\
	AXP & AIG & AMT & AWK & AMP & ABC\\
	AME & AMGN & APH & ADI & ANSS & ANTM\\
	AON & AOS & APA & AIV & AAPL & AMAT\\
	APTV & ADM & ANET & AJG & AIZ & T\\
	ATO & ADSK & ADP & AZO & AVB & AVY\\
	BKR & BLL & BAC & BK & BAX & BDX\\
	BRK.B & BBY & BIO & BIIB & BLK & BA\\
	BKNG & BWA & BXP & BSX & BMY & AVGO\\
	BR & BF.B & CHRW & COG & CDNS & CPB\\
	COF & CAH & KMX & CCL & CARR & CTLT\\
	CAT & CBOE & CBRE & CDW & CE & CNC\\
	CNP & CERN & CF & SCHW & CHTR & CVX\\
	CMG & CB & CHD & CI & CINF & CTAS\\
	CSCO & C & CFG & CTXS & CLX & CME\\
	CMS & KO & CTSH & CL & CMCSA & CMA\\
	CAG & CXO & COP & ED & STZ & COO\\
	CPRT & GLW & CTVA & COST & CCI & CSX\\
	CMI & CVS & DHI & DHR & DRI & DVA\\
	DE & DAL & XRAY & DVN & DXCM & FANG\\
	DLR & DFS & DISCA & DISCK & DISH & DG\\
	DLTR & D & DPZ & DOV & DOW & DTE\\
	DUK & DRE & DD & DXC & EMN & ETN\\
	EBAY & ECL & EIX & EW & EA & EMR\\
	ETR & EOG & EFX & EQIX & EQR & ESS\\
	EL & ETSY & EVRG & ES & RE & EXC\\
	EXPE & EXPD & EXR & XOM & FFIV & FB\\
	FAST & FRT & FDX & FIS & FITB & FE\\
	FRC & FISV & FLT & FLIR & FLS & FMC\\
	F & FTNT & FTV & FBHS & FOXA & FOX\\
	BEN & FCX & GPS & GRMN & IT & GD\\
	GE & GIS & GM & GPC & GILD & GL\\
	GPN & GS & GWW & HAL & HBI & HIG\\
	HAS & HCA & PEAK & HSIC & HSY & HES\\
	HPE & HLT & HFC & HOLX & HD & HON\\
	HRL & HST & HPQ & HUM & HBAN & HII\\
	IEX & IDXX & INFO & ITW & ILMN & INCY\\
	IR & INTC & ICE & IBM & IP & IPG\\
	IFF & INTU & ISRG & IVZ & IPGP & IQV\\
	IRM & JKHY & J & JBHT & SJM & JNJ\\
	JCI & JPM & JNPR & KSU & K & KEY\\
	KEYS & KMB & KIM & KMI & KLAC & KHC\\
	KR & LB & LHX & LH & LRCX & LW\\
	LVS & LEG & LDOS & LEN & LLY & LNC\\
	LIN & LYV & LKQ & LMT & L & LOW\\
	LYB & MTB & MRO & MPC & MKTX & MAR\\
	MMC & MLM & MAS & MA & MKC & MXIM\\
	MCD & MCK & MDT & MRK & MET & MTD\\
	MGM & MCHP & MU & MSFT & MAA & MHK\\
	\hline
\end{tabular}
\end{table}

\begin{table}[htbp]
\caption{Tickers (continued) of the GMVP portfolio}
\label{table:tickersCont}
\centering
\begin{tabular}{ |C{2cm}|C{2cm}|C{2cm}|C{2cm}|C{2cm}|C{2cm}|  }
	\hline
	TAP & MDLZ & MNST & MCO & MS & MOS\\
	MSI & MSCI & MYL & NDAQ & NOV & NTAP\\
	NFLX & NWL & NEM & NWSA & NWS & NEE\\
	NLSN & NKE & NI & NSC & NTRS & NOC\\
	NLOK & NCLH & NRG & NUE & NVDA & NVR\\
	ORLY & OXY & ODFL & OMC & OKE & ORCL\\
	OTIS & PCAR & PKG & PH & PAYX & PAYC\\
	PYPL & PNR & PBCT & PEP & PKI & PRGO\\
	PFE & PM & PSX & PNW & PXD & PNC\\
	POOL & PPG & PPL & PFG & PG & PGR\\
	PLD & PRU & PEG & PSA & PHM & PVH\\
	QRVO & PWR & QCOM & DGX & RL & RJF\\
	RTX & O & REG & REGN & RF & RSG\\
	RMD & RHI & ROK & ROL & ROP & ROST\\
	RCL & SPGI & CRM & SBAC & SLB & STX\\
	SEE & SRE & NOW & SHW & SPG & SWKS\\
	SLG & SNA & SO & LUV & SWK & SBUX\\
	STT & STE & SYK & SIVB & SYF & SNPS\\
	SYY & TMUS & TROW & TTWO & TPR & TGT\\
	TEL & FTI & TDY & TFX & TER & TXN\\
	TXT & TMO & TIF & TJX & TSCO & TT\\
	TDG & TRV & TFC & TWTR & TYL & TSN\\
	UDR & ULTA & USB & UAA & UNP & UAL\\
	UNH & UPS & URI & UHS & UNM & VFC\\
	VLO & VAR & VTR & VRSN & VRSK & VZ\\
	VRTX & V & VNO & VMC & WRB & WAB\\
	WMT & WBA & DIS & WM & WAT & WEC\\
	WFC & WELL & WST & WDC & WU & WRK\\
	WY & WHR & WMB & WLTW & WYNN & XEL\\
	XRX & XLNX & XYL & YUM & ZBRA & ZBH\\
	ZION & ZTS & & & & \\
	\hline
\end{tabular}
\end{table}

\bibliography{references}
% PDFLaTeX + MakeIndex + BibTeX

\end{document}
 
