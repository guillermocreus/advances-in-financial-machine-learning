\chapter{Introduction}

Since the main goal of the thesis is to explore several advances of Machine 
Learning in Finance, three techniques firstly introduced by Marcos López de 
Prado \cite{AdvFML} have been chosen. The methods are the following:

\begin{enumerate}
	\item Meta-labeling
	\item Fractional differentiation
	\item Data parsing as bars
\end{enumerate}

While meta-labeling is a finance-specific modification of binary 
classification, the last two methods deal mostly with data 
modification/pre-processing. Although the methods are different in essence, 
the objective is similar. That is, compare the methods to standard 
techniques so as to determine if they provide an edge to practitioners.\\

When dealing with financial data, where signal-to-noise ratio is low, 
establishing the validity of the previous statement is not clear. That is 
why the first two methods have been validated with synthetic data in what 
will be called ``Toy Projects''. These projects will be extremely important 
because they will provide a controlled environment where statistical 
parameters can be tuned to our liking, helping to determine the true 
performance of the proposed method.

\section{Document overview}
The work will be structured in the following chapters:

\begin{itemize}
	\item Chapter \ref{chapterIntroFinDat} will introduce readers to
	essential financial concepts that will be extremely vital to understand
	the succeeding chapters.
	
	\item Chapter \ref{chapterMetaLabeling} explores the concept of 
	Meta-labeling via a Toy Project to later apply it to financial data.
	
	\item Chapter \ref{chapterFracDiff} will present fractional 
	differentiation of time series in order to obtain stationary time series 
	without giving up memory. Again, before applying the method to financial 
	time series, a Toy Project will be created.
	
	\item Chapter \ref{chapterDataParsing} will explain a novel way to 
	sample high frequency data in finance. Also, it will illustrate how it 
	can improve forecasts.
	
	\item Chapter \ref{chapterConclFutWork} shows the impact these 
	techniques could have in Finance, analyzes the standing of Machine 
	Learning in Finance and reflects on future possibilities that could 
	arise from this work.
\end{itemize}

