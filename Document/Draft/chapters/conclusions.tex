\chapter{Conclusions and future work}
\label{chapterConclFutWork}
% Paragraph 1
% Overview of the methods - connection with the abstract
The goal of this thesis was to explore meta-labeling, fractional 
differentiation and data parsing as bars in the financial context with the 
purpose of determining if they were useful at delivering better forecasts 
and/or higher risk-adjusted returns. In order to fulfill this goal, three 
independent chapters have been presented, which have analyzed the techniques 
mentioned.\\

% Paragraph 2
% Conclusion of each method
First of all, after studying meta-labeling thoroughly it can be concluded 
that this technique is very specific in the sense that it needs a certain 
environment to deliver better results. To be precise, as it was seen via the 
\textit{coin flip} correction, to deliver higher Sharpe Ratios it needed a 
primary model that was good by itself, Sharpe Ratio wise. As the primary 
models developed did not achieve satisfactory Sharpe Ratios, the results 
mentioned before were not realized. Therefore, meta-labeling should only be 
used if one is confident that their primary model is good but needs minimal 
tweaking.\\

Secondly, fractional differentiation was useful at providing a framework to 
achieve stationarity without losing memory in time series. However, it did 
not translate into better forecasts. This is attributed to financial data, 
where memory was determined to not be highly relevant at giving predictions.
\\

Lastly, the sampling technique known as data parsing as bars was helpful at 
giving better forecasts. It evidenced that in high frequency data, activity 
is more important than time.\\

% Paragraph 3
Furthermore, it should be pointed out that there is still room for 
improvement. The data used in this thesis was centered around time series. 
However, in Finance there are alternative data sources such as financial 
news, macroeconomic data, FOREX, etc. It remains to be seen how these could 
have improved the performance of the Machine Learning models.\\

% Paragraph 4
Regarding future work that might spin off of this thesis, it is important to 
mention the design of high frequency trading strategies. Having explored 
sampling of high frequency data, it is natural to continue working with this 
type of data, which is the perfect environment to further test 
meta-labeling, fractional differentiation ...\\

% Paragraph 5
On the personal side, the author has learned that using Machine Learning in 
Finance is not a matter of what but when. One can have the most promising 
techniques but fail to deliver results due to not applying the model to the 
correct data set. In part, this is what happened in the meta-labeling and 
fractional differentiation chapters, which in financial data failed to 
deliver the results observed in the toy projects.\\

% Paragraph 6
As a final note, it should be acknowledged that this thesis has been 
extremely useful to learn concepts that the author was oblivious of. That is, 
time series analysis, high frequency financial data and Machine Learning 
models. Also, this work has been an excellent introduction to academic 
research, since it has laid the foundations of the author's academic career, 
which he intends to continue in the coming years.