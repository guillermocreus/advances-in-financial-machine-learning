\thispagestyle{empty}
\begin{center}
	\Large
	\textbf{Exploring Machine Learning Advances in Finance}
	
	\large
	\vspace{.4cm}
	\textit{by} Guillermo Creus Botella
	
	\vspace{.9cm}
	\textbf{Abstract}
\end{center}

It is not a mystery that Machine Learning has revolutionized the way humans 
analyze data. However, when treating data as complex as the one found in the 
stock market, clear advances are yet to be accomplished. This work will be 
focused on three novel areas of Machine Learning applied to finance: meta-
labeling, fractional differentiation and financial data parsing in the form 
of bars. Every aspect will be analyzed independently so as to determine its 
individual effectiveness.\\

Furthermore, as Machine Learning thrives in data, one should be very careful 
with the information provided to algorithms. That is why, whenever possible, 
the proposed solution will be tested with synthetic data, providing a 
controlled environment. When success is achieved under these conditions, 
the procedure will be put to use with real data, where these methods will be 
compared with standard techniques to ascertain if predictions deliver better 
forecasts or strategies with higher risk-adjusted returns.\\

\textbf{Keywords:} Machine Learning, quantitative finance, Meta-labeling, 
fractional differentiation, data parsing, time series analysis, feature 
engineering.\\

\textbf{MSC Code:} 91B84

\newpage
\thispagestyle{empty}
\begin{center}
	\Large
	\textbf{Explorando avances del aprendizaje automático en las finanzas}
	
	\large
	\vspace{.4cm}
	\textit{por} Guillermo Creus Botella
	
	\vspace{.9cm}
	\textbf{Resumen}
\end{center}

No es un misterio que el aprendizaje automático ha revolucionado la forma en 
que los humanos analizamos datos. Sin embargo, al tratar datos tan complejos 
como los que se encuentran en el mercado de valores, aún no se han logrado 
avances claros. Este trabajo estará centrado en tres áreas novedosas del 
aprendizaje automático aplicado a las finanzas: Meta-labeling, 
diferenciación fraccional y \textit{parsing} de datos en forma de barras. 
Cada aspecto será analizado de forma independiente para determinar su 
eficacia individual.\\

Además, dado que el aprendizaje automático se nutre de datos, se debe tener 
mucho cuidado con la información proporcionada a los algoritmos usados. Por 
eso, siempre que sea posible, la solución propuesta se probará con datos 
sintéticos, proporcionando un ambiente controlado. Cuando se logre el éxito 
en estas condiciones, se utilizará el procedimiento con datos reales, donde 
se compararán los métodos usados con técnicas estándar para determinar si 
las predicciones ofrecen mejores pronósticos o estrategias con mayores
rendimientos ajustados al riesgo.\\


\textbf{Palabras clave:} Aprendizaje automático, matemática financiera, 
Meta-labeling, diferenciación fraccional, \textit{parsing} de datos, 
análisis de series temporales, \textit{feature engineering}.\\

\textbf{Código MSC:} 91B84

\newpage
\thispagestyle{empty}
\begin{center}
	\Large
	\textbf{Explorant avenços d'aprenentatge automàtic a les finances}
	
	\large
	\vspace{.4cm}
	\textit{per} Guillermo Creus Botella
	
	\vspace{.9cm}
	\textbf{Resum}
\end{center}

No és un misteri que l'aprenentatge automàtic ha revolucionat la forma en
que els humans analitzem dades. No obstant, a l'hora de tractar dades tan 
complexes com les trobades al mercat de valors, encara no s'han aconseguit
avenços clars. Aquest treball estara centrat en tres àrees noves de 
l'aprenentatge automàtic aplicat a les finances: Meta-labeling, 
diferenciació fraccional i \textit {parsing} de dades en forma de barres. 
Cada aspecte serà analitzat de forma independent per determinar la seva 
eficàcia individual.\\

A més, atès que l'aprenentatge automàtic es sustenta amb dades, s'ha d'anar 
amb molt de compte amb la informació proporcionada als algoritmes usats. Per
això, sempre que sigui possible, la solució proposada es provarà amb dades
sintètiques, proporcionant un ambient controlat. Un cop assolit l'èxit en 
aquestes condicions, s'utilitzarà el procediment amb dades reals, on es 
compararan els mètodes usats amb tècniques estàndard per determinar si les 
prediccions donen millors pronòstics o estratègies amb majors rendiments 
ajustats al risc.\\

\textbf{Paraules clau:} Aprenentatge automàtic, matemàtica financera, 
Meta-labeling, diferenciació fraccional, \textit{parsing} de dades, anàlisi 
de sèries temporals, \textit{feature engineering}.\\

\textbf{Codi MSC:} 91B84